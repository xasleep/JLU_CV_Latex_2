% XeLaTeX or LuaLaTeX needed to compile due to fontspec package
% 编译本文档需要使用 XeLaTeX 或 LuaLaTeX 编译器,因为使用了 fontspec 宏包

\documentclass[11pt]{article} % 定义文档类型为 article,全局默认字号为 11pt
                               % 可选:10pt, 12pt。article 是标准文档类。

%%%%%%%%%%%%%%%%%%%%
% 引入宏包 (Packages)
%%%%%%%%%%%%%%%%%%%%

\usepackage{hyperref}       % 用于创建超链接 (例如网址、邮箱、内部引用)
\usepackage{xcolor}         % 提供颜色支持,可以定义和使用颜色
\usepackage{calc}           % 允许在 LaTeX 参数中进行简单的数学计算 (如长度加减)
\usepackage{graphicx}       % 用于插入图片 (命令: \includegraphics)
\usepackage{tikz}           % 强大的绘图宏包,用于创建复杂的图形,这里用于页眉页脚和背景
\usepackage{fontspec}       % (仅限 XeLaTeX/LuaLaTeX) 允许方便地使用系统安装的 OpenType/TrueType 字体,特别是中文字体
\usepackage{fontawesome5}   % 提供 Font Awesome 5 图标集 (如邮箱、电话、GitHub图标)
\usepackage{titlesec}       % 用于高度自定义章节标题 (section, subsection 等) 的格式
\usepackage{enumitem}       % 提供对列表环境 (itemize, enumerate) 的精细控制 (间距、标签等)
\usepackage{fancybox}       % 提供一些花哨的盒子命令 (如 \doublebox 给图片加双线框)

% 设置 hyperref 宏包
\hypersetup{hidelinks}      % 取消链接的边框和特殊颜色,使其看起来更自然
                            % 可选:移除此行或设置 colorlinks=true, linkcolor=blue 等来显示链接颜色

%%%%%%%%%%%%%%%%%%%%
% 全局设置 (Global Settings)
%%%%%%%%%%%%%%%%%%%%

\setlength{\parindent}{0pt} % 设置全局段落首行缩进为 0pt (即不缩进)。段落间通常通过 \parskip (段间距) 区分。
                            % 可选:设置为 2em (约两个大写M的宽度) 或其他你希望的缩进值。

\pagenumbering{gobble}      % 取消页码显示。"gobble" 意味着 "吞掉" 页码。
                            % 可选:改为 "arabic" (1, 2, 3...) 或其他样式以显示页码。简历通常不显示。

% 自定义 itemize (无序列表) 样式
\setlist[itemize]{
    nosep                   % 移除列表项之间及列表与前后文本的垂直间距,使列表更紧凑。
    , before={\vspace*{-\parskip}} % 移除 itemize 环境前可能因 \parskip 非零而产生的额外空白。
    , leftmargin=*          % 设置列表的左边距。'*' 表示使用一个合理的默认值,使项目符号靠近页边。
                            % 可选:leftmargin=2em (增加左缩进), itemsep=0.5ex (设置项间距) 等。
}

% 自定义 enumerate (有序列表) 样式
\setlist[enumerate]{
    leftmargin=*            % 设置列表的左边距,同 itemize 中的 leftmargin=*。
}

\renewcommand{\arraystretch}{1.2} % 设置表格中行高的倍数。1.2 表示行高为默认的1.2倍,增加垂直空间。
                                 % 可选:1.0 (默认), 1.5 (更大间距) 等。

\linespread{1.25}           % 设置正文的行间距。1.25 大约相当于 1.25 倍行距。
                            % 可选:1.0 (单倍行距), 1.5 (1.5倍行距) 等。简历推荐略大于1以便阅读。

% 自定义 \section (一级标题) 的格式 (使用 titlesec 宏包)
\titleformat{\section}      % 命令作用于 \section
  {\LARGE\bfseries\raggedright} % 标题文本的格式:
                                % \LARGE: 字体大小设为 \LARGE (比 \Large 大)
                                % \bfseries: 粗体
                                % \raggedright: 左对齐
                                % 可选:\Large, \huge (字号); \mdseries (正常字重); \centering (居中)
  {}                          % 标题的标签 (如章节号)。这里为空,表示不显示章节号。
  {0em}                       % 标签和标题文本之间的水平间距。
  {}                          % 在标题文本之前插入的代码。这里为空。
  [{\color{secondary_color}\titlerule}] % 在标题文本之后 (另起一行) 插入的代码:
                                % \color{secondary_color}: 将颜色设为 secondary_color
                                % \titlerule: titlesec 提供的命令,画一条默认粗细的横线
                                % 可选:修改颜色, \titlerule[0.8pt] (改变线粗), 或去掉此行取消下划线

% 自定义 \section 标题的间距 (使用 titlesec 宏包)
% \titlespacing*{<command>}{<left>}{<before-sep>}{<after-sep>}
% 星号版本表示 <left> 会覆盖段落缩进
\titlespacing*{\section}
  {0cm}                       % 标题左边的间距。
  {*1.2}                      % 标题上方的垂直间距。'*1.2' 表示 1.2 倍的 \baselineskip (当前行基线间距)。
  {*1.2}                      % 标题下方的垂直间距 (标题和正文第一行之间)。
                              % 可选:调整这些值来改变标题周围的空白。

% 页面边距设置 (使用 geometry 宏包)
\usepackage[
	a4paper,                    % 纸张类型为 A4
	left=1.2cm,                 % 左边距
	right=1.2cm,                % 右边距
	top=1.5cm,                  % 上边距
	bottom=1cm,                 % 下边距
	nohead                      % 不为页眉预留标准空间 (因为使用 TikZ 自定义页眉)
]{geometry}
% 可选:根据需要调整各边距值。

% 字体设置 (使用 fontspec 宏包 - 需要 XeLaTeX/LuaLaTeX)
\setmainfont[                 % 设置文档的主要字体 (用于西文和中文,如果字体支持)
    Path=fonts/,              % LaTeX 将在当前目录下的 "fonts/" 文件夹中寻找字体文件
                              % 可选:如果字体已安装到系统,此行可省略或修改路径。
    Extension=.otf,           % 字体文件的扩展名 (如 .otf, .ttf)
    BoldFont=*-Bold,          % 指定粗体变体。'*' 代表主字体名 (NotoSerifSC),所以会寻找 NotoSerifSC-Bold.otf
                              % 可选:根据实际粗体文件名修改,如 BoldFont=*Bold (如果文件名是 NotoSerifSCBold.otf)
    % ItalicFont=*-Italic,    % (可选) 指定斜体变体
    % BoldItalicFont=*-BoldItalic % (可选) 指定粗斜体变体
]{NotoSerifSC}                % 主字体名称 (这里是思源宋体 SC 版)
                              % 可选:替换为你喜欢的其他字体名称,确保字体文件可用。

% 自定义颜色 (使用 xcolor 宏包)
% 使用 RGB 模型定义颜色,值范围 0-255
\definecolor{primary_color}{RGB}{74,164,132}    % 定义主颜色 "取色器南师绿"
                                                % 可选:修改 RGB 值或使用其他颜色模型 (如 HTML)
\definecolor{secondary_color}{RGB}{74,164,132} % 定义次要颜色,当前与主色相同。
                                                % 可选:修改RGB值以使用不同颜色。

% 自定义长度变量
\newlength{\iconwidth}      % 声明一个新的长度变量 \iconwidth
\setlength{\iconwidth}{1.5em} % 设置 \iconwidth 的值为 1.5em。
                              % 'em' 是相对单位,约等于当前字体大写 "M" 的宽度。用于 section 标题图标占位。
                              % 可选:调整此值以适应图标大小或对齐需求。

%%%%%%%%%%%%%%%%%%%%
% 自定义命令 (Custom Commands)
%%%%%%%%%%%%%%%%%%%%

% 学院信息
\newcommand{\school}{地理科学学院 | School of Geography} % 定义 \school 命令存储学院信息

% 联系方式
\newcommand{\contact}{%
    % 根据个人喜好选择字号
    \small                % 小号字 (当前选择)
    % \footnotesize           % 更小号字
    % \scriptsize           % 再小一号字
    \textcolor{black}{%     % 设置联系方式文本颜色为黑色
                            % 可选:若页脚背景为深色,可改为 \textcolor{white}
        % 邮箱
        \faEnvelope \quad \href{mailto:youremail@njnu.edu.cn}{youremail@njnu.edu.cn} % \faEnvelope: 邮箱图标; \quad: 水平空白; \href: 创建链接
                                                                                  % 邮箱后缀已更新
        \hspace{4em}        % 创建一个 4em 宽的水平空白,用于分隔联系方式条目
                            % 可选:调整 \hspace 的值来改变间距
        % 手机号
        \faPhone \quad  130-xxxx-xxxx % \faPhone: 电话图标
        % 别的联系方式,如微信、GitHub等
        \hspace{4em}
        \faGithub \quad \href{https://github.com/Reborn14}{GitHub 项目地址} % \faGithub: GitHub图标
                            % 可选:替换为你自己的信息、图标和链接,增删条目
    }%
}

%%%%%%%%%%%%%%%%%%%%
% 文档正文开始
%%%%%%%%%%%%%%%%%%%%
\begin{document}

    %%%%%%%%%%%%%%%%%%%%
    % 页眉、页脚和背景图 (使用 TikZ)
    % 注意:这些 TikZ 环境使用 remember picture 和 overlay 选项,
    % 它们会在文档编译两次后才能正确定位。
    % 如果简历有多页,理想情况下应使用 fancyhdr 或 eso-pic 等宏包自动处理每页的页眉页脚,
    % 而不是手动复制粘贴这部分代码。当前模板是为单页优化的。
    %%%%%%%%%%%%%%%%%%%%

    % --- 页眉:校标组合+学院名 ---
    \begin{tikzpicture}[remember picture, overlay] % remember picture: 允许跨 TikZ 图引用节点; overlay: 内容不占空间,浮动在文本之上
        % 页眉背景图
        \node[anchor=north,         % 节点的锚点在其上边缘中点
              inner sep=0pt,        % 节点内容与边框的内部间距为0
             ]
             (header)               % 给这个节点命名为 "header"
             at (current page.north)% 定位节点的锚点在当前页面的顶部中点
             {\includegraphics[width=\paperwidth]{images/header.png}}; % 插入与页面同宽的图片作为页眉背景
                                                                    % 图片路径已更新为 images/header.png

        % 校徽/学校标识 (放置在页眉背景图的左侧)
        \node[anchor=west]          % 节点的锚点在其左边缘中点
             (school_logo)          % 命名节点
             at (header.west)       % 定位在 "header" 节点的左边缘中点
             {
                \hspace{0.5cm}      % 向右推移校徽,产生左边距
                \includegraphics[width=0.3\textwidth]{images/header_nnu.png} % 插入校徽图片,宽度为文本区域的30%
                                                                          % 图片路径已更新为 images/header_nnu.png, 宽度为0.3\textwidth
             };

        % 学院名称 (放置在页眉背景图的右侧)
        \node[anchor=east]          % 节点的锚点在其右边缘中点
             (school_name)          % 命名节点
             at(header.east)        % 定位在 "header" 节点的右边缘中点
             {
                \textcolor{white}{\large\textbf{\school}} % 显示 \school 命令的内容 (学院名),白色,粗体,大号
                \hspace{0.5cm}      % 向左推移学院名,产生右边距
             };
    \end{tikzpicture}
    \vspace{-4.5em}                 % 负垂直间距,将后续内容向上拉。值已更新为 -4.5em。
                                    % 注意:此值可能需要根据页眉图片高度、字体、行距等进行微调。

    % --- 页脚:联系方式 ---
    \begin{tikzpicture}[remember picture, overlay]
        % 页脚背景图
        \node[anchor=south,         % 节点的锚点在其下边缘中点
              inner sep=0pt
             ]
             (footer)               % 命名节点 "footer"
             at (current page.south)% 定位在当前页面的底部中点
             {\includegraphics[width=1.1\paperwidth, height=1cm]{images/foot.png}}; % 插入页脚背景图
                                                                    % 图片路径 images/foot.png
                                                                    % width=1.1\paperwidth: 图片宽度为页面宽度的110%,会超出页面。
                                                                    % height=1cm: 图片高度强制为1cm,可能导致图片变形。
                                                                    % 注意:原代码中此行末尾的特殊空白字符已移除。

        % 联系方式 (放置在页脚背景图的中心)
        \node[anchor=center]        % 节点的锚点在其中心
             at(footer.center)      % 定位在 "footer" 节点的中心
             {\contact};            % 显示 \contact 命令的内容 (联系方式)
    \end{tikzpicture}

    % --- 背景水印 ---
    \begin{tikzpicture}[remember picture, overlay]
        \node[opacity=0.05]         % 设置节点 (图片) 的透明度为 5% (0.0 到 1.0)
             at(current page.center)% 定位在当前页面的中心
             {
                \includegraphics[width=0.7\paperwidth, % 图片宽度为页面宽度的70%
                                     keepaspectratio % 保持图片的原始宽高比
                                    ]{images/nnu_logo.png} % 背景水印图片                                                          
             };
    \end{tikzpicture}

    %%%%%%%%%%%%%%%%%%%%
    % 简历正文 (Main Content)
    %%%%%%%%%%%%%%%%%%%%

    % --- 主内容区:左侧78%宽度,右侧20%宽度给照片 ---
    % 使用 minipage 环境创建左右分栏效果
    \begin{minipage}[t]{0.78\textwidth} % 左栏 minipage,宽度为文本区域的78%,[t]表示顶部对齐
        % --- 个人信息 ---
        \begin{minipage}[t]{\textwidth} % 个人信息部分占满其父 minipage 的宽度
        \section[个人信息]{\makebox[\iconwidth][c]{\color{primary_color}{\faAddressCard}}\quad 个人信息}
            % \section[可选短标题]{标题内容} : [可选短标题] 用于生成目录,简历中一般可省略。
            % \makebox[\iconwidth][c]{...}: 创建一个固定宽度 (\iconwidth) 的盒子,内容在其中居中 [c]。用于放置图标。
            % \color{primary_color}{\faAddressCard}: 设置图标颜色并插入图标。
            % \quad: 图标和文字之间的固定间距。

            % 个人信息内部再分两栏
            \begin{minipage}[t]{0.5\textwidth} % 个人信息左半部分
                \textbf{姓\qquad 名}:你的名字 % \qquad 是一个较宽的固定空白,用于对齐冒号

                \vspace{0.5em} % 垂直间距
                \textbf{出生年月}:你的出生年月
            \end{minipage}% 注意这里的 % 号,用于避免 minipage 之间产生不必要的空格
            \begin{minipage}[t]{0.35\textwidth} % 个人信息右半部分
                \textbf{性\qquad 别}:你的性别
                
                \vspace{0.5em}
                \textbf{政治面貌}:你的政治面貌
            \end{minipage}
            \vspace{1.2em} % 个人信息区块下方的垂直间距
        \end{minipage}

        % --- 教育背景 ---
        \begin{minipage}[t]{\textwidth}
        \section[教育背景]{\makebox[\iconwidth][c]{\color{primary_color}{\faGraduationCap}}\quad 教育背景}

        {\large \textbf{南京师范大学}},专科 \hfill 2000年9月--2010年6月
        \begin{itemize} % 无序列表
            \item 你的学院,你的专业
            \item \textbf{主修课程}:课程1、课程2、课程3、课程4\ 等。
        \end{itemize}

        \vspace{0.5em} % 条目间垂直间距
        {\large \textbf{南京师范大学}},本科 \hfill 2010年9月--2020年6月
        \begin{itemize}
            \item 你的学院,你的专业
            \item \textbf{主修课程}:课程1、课程2、课程3、课程4\ 等。
            \item \textbf{GPA}:4.8 / 4.8(排名:1 / 250)
        \end{itemize}

        \vspace{0.5em}
        {\large \textbf{南京师范大学}},硕士 \hfill 2020年9月--至今
        \begin{itemize}
            \item 你的学院,你的专业,你的导师姓名\ 职称
            \item \textbf{研究方向}:方向1、方向2、方向3、方向4\ 等。
        \end{itemize}

        \vspace{1.2em} % 教育背景区块下方的垂直间距
        \end{minipage}
    \end{minipage} % 结束左侧主内容区的 minipage
    \hfill % 水平填充空白,将右侧 minipage 推到最右边
    % --- 右侧边栏:照片 ---
    \begin{minipage}[t]{0.2\textwidth} % 右栏 minipage,宽度为文本区域的20%,[t]表示顶部对齐
        \vspace{2em} % 照片上方的垂直空白,用于调整照片的垂直位置
        \setlength{\fboxsep}{0pt} % 设置 \fbox (以及 \doublebox) 内容与边框的间距为 0pt。
        \doublebox{\includegraphics[width=\linewidth]{images/avatar.png}} % \doublebox: fancybox宏包提供的双线框
                                                                    % \includegraphics[width=\linewidth]: 图片宽度等于当前 minipage 的宽度
    \end{minipage} % 结束右侧照片栏的 minipage

    % --- 科研成果 --- (此 minipage 跨越整个文本宽度)
    \begin{minipage}[t]{\textwidth}
    \section[科研成果]{\makebox[\iconwidth][c]{\color{primary_color}{\faAtom}}\quad 科研成果}

    % 科研著作(研究生)
    This is One of Your Paper Published in Conference A. % 论文标题或描述
    \begin{itemize}
        \item \textbf{Mingzi Nide}, Daoshi Nide. \hfill 发表于 \textbf{Conference A}(CCF-A类会议) % 作者,发表信息
        \item 用一句话描述这份论文干了什么\dots\dots % 论文简介
    \end{itemize}

    \vspace{0.5em}
    《另一份论文的标题》 % 论文标题
    \begin{itemize}
        \item  \textbf{你的名字}、你的师兄、你的导师 \hfill 发表于 \textbf{某篇期刊} (SCI-1区) % 作者,发表信息
        \item 用一句话描述这份论文干了什么\dots\dots % 论文简介
    \end{itemize}

    \vspace{1.2em} % 科研成果区块下方的垂直间距
    \end{minipage}

    % --- 项目经历/科研经历/项目与教学 --- (此 minipage 跨越整个文本宽度)
    \begin{minipage}[t]{\textwidth}
    % 标题请根据需要修改
    \section[项目与教学]{\makebox[\iconwidth][c]{\color{primary_color}{\faChalkboardTeacher}}\quad 项目与教学}

    {\large \textbf{项目名称}} \hfill 2020年9月--2021年9月 % 项目名,时间
    \begin{itemize}
        \item \textbf{你在项目中扮演的角色} \hfill 横向/纵向项目-已完结/进行中 % 角色,项目类型/状态
        \item 用一句话介绍你在这个项目中做了什么\dots\dots % 项目描述
    \end{itemize}

    \vspace{0.5em}
    {\large \textbf{某某主题讨论班}},主讲 / 参与 \hfill 2020年夏季 % 活动名,角色,时间
    \begin{itemize}
        \item \textbf{主要内容}:内容1,内容2,内容3\ 等。 % 内容描述
    \end{itemize}

    \vspace{0.5em}
    {\large \textbf{课程名称}},助教 \hfill 2021年夏季 % 课程名,角色,时间
    \begin{itemize}
        \item \textbf{主要内容}:内容1,内容2,内容3\ 等。 % 内容描述
    \end{itemize}

    \vspace{1.2em} % 项目经历区块下方的垂直间距
    \end{minipage}

    % --- 技能特长与兴趣爱好并排显示 ---
    % 如果每行的内容不是很多,可以考虑使用 minipage,将内容分列展示
    \begin{minipage}[t]{0.6\textwidth} % 左侧技能特长栏,宽度为文本区域的60%
        \section[技能特长]{\makebox[\iconwidth][c]{\color{primary_color}{\faWrench}}\quad 技能特长}
        \begin{itemize}
        \setlength{\itemsep}{0.5em} % 在此 itemize 环境内部设置列表项之间的垂直间距为 0.5em
            \item 熟练使用 Python、Jvav、Rust 等编程语言。 % 可选:替换为你的技能
            \item 熟练使用 Tensorflow、Pytorch 等深度学习框架。
            \item 熟悉 Windows 与 Linux 端开发。
        \end{itemize}
    \end{minipage}%
    \hfill % 水平填充,将右侧 minipage 推开
    \begin{minipage}[t]{0.35\textwidth} % 右侧兴趣爱好栏,宽度为文本区域的35%
        \section[兴趣爱好]{\makebox[\iconwidth][c]{\color{primary_color}{\faStar}}\quad 兴趣爱好}
        \begin{itemize}
        \setlength{\itemsep}{0.5em} % 同上,设置项间距
            \item 爱好1 % 可选:替换为你的爱好
            \item 爱好2
            \item 爱好3
            \item 爱好4
        \end{itemize}
    \end{minipage}

    % --- (可选) 更多页面和内容 ---
    % \newpage % 如果需要新的一页,取消此行注释
    % % 如有需要,可以添加额外的页面。
    % % 不要忘记添加页眉页脚和背景相关的代码 (如果未使用自动化方案如 fancyhdr/eso-pic)。

    % % --- (示例) 竞赛经历 ---
    % \section{\makebox[\widthof{\faTrophy}][c]{\color{primary_color}{\faTrophy}}\quad 竞赛经历}
    %     % \widthof{\faTrophy} 会自动计算 \faTrophy 图标的宽度,用于 \makebox。
    %     % 这是一个更动态的设置图标占位宽度的方法,相比固定的 \iconwidth。
    % \begin{table}[h!] % 表格环境,[h!] 表示 LaTeX "尽量"将表格放在此处
    %     \begin{tabularx}{\textwidth}{Xp{\widthof{第零负责人}}p{\widthof{国家级-第100名}}p{\widthof{2030年13月}}}
    %         % tabularx 环境,总宽度为 \textwidth。
    %         % X: 可伸缩列,自动填充剩余宽度。
    %         % p{width}: 固定宽度的段落列,内容会自动换行。
    %         % \widthof{text}: 计算 "text" 的宽度,用于精确控制列宽。
    %         \textbf{比赛1} & 第一负责人 & 国家级-第10名 & 2023年4月 \\
    %         \textbf{比赛2} & 个人参赛 & 国家级-一等奖 & 2023年8月\\
    %         \textbf{比赛3} & 个人参赛 & 省级-一等奖 & 2022年12月\\
    %         % 同理,可以自己加
    %     \end{tabularx}
    % \end{table}

    % % --- (示例) 技能特长 (如果想用不同格式或放在第二页) ---
    % \section{\makebox[\widthof{\faWrench}][c]{\color{primary_color}{\faWrench}}\quad 技能特长}
    % \begin{itemize}
    %     \item 熟练使用\Cpp 、Python、Matlab编程语言。
    %     \item 熟悉Windows与Linux端开发。
    %     % ... (更多技能)
    % \end{itemize}

    % % --- (示例) 所获荣誉 ---
    % \usepackage{multicol} % 如果要使用 multicols 环境,需要在导言区 \usepackage{multicol}
    % \section{\makebox[\widthof{\faStar}][c]{\color{primary_color}{\faStar}}\quad 所获荣誉}
    % \begin{multicols}{2} % 将内容分为两栏
    %     \begin{itemize}
    %         \item 某年学业先进个人
    %         \item 某年某奖学金某等奖
    %         % ... (更多荣誉)
    %     \end{itemize}
    % \end{multicols}

    % % --- (示例) 其他 ---
    % \section{\makebox[\widthof{\faInfo}][c]{\color{primary_color}{\faInfo}}\quad 其他}
    % \begin{itemize}
    %     \item 英语水平-CET6级xxx分
    %     \item 计算机几级证书
    %     % ... (其他信息)
    % \end{itemize}

\end{document}