\documentclass[11pt]{article}

\usepackage{hyperref}
\usepackage{xcolor}
\usepackage{calc}
\usepackage{graphicx}
\usepackage{tikz}
\usepackage{fontspec}
\usepackage{fontawesome5}
\usepackage{titlesec}
\usepackage{enumitem}
\usepackage{fancybox}
\usepackage{multicol}

\hypersetup{hidelinks}
\definecolor{navyblue}{RGB}{0,0,128}  % 标准海军蓝 RGB(0,0,128)
%%%%%%%%%%%%%%%%%%%%
% 设置
%%%%%%%%%%%%%%%%%%%%

\setlength{\parindent}{0pt}
\pagenumbering{gobble}
\setlist[itemize]{nosep
	, before={\vspace*{-\parskip}}
	, leftmargin=*}
\setlist[enumerate]{leftmargin=*}
\renewcommand{\arraystretch}{1.2}
\linespread{1.25}

\titleformat{\section}
{\LARGE\bfseries\raggedright}
{}{0em}
{}
[{\color{secondary_color}\titlerule}]
\titlespacing*{\section}{0cm}{*1.2}{*1.2}

\usepackage[
a4paper,
left=1.2cm,
right=1.2cm,
top=1.5cm,
bottom=1cm,
nohead
]{geometry}

\setmainfont[
Path=fonts/,
Extension=.otf,
BoldFont=*-Bold,
]{NotoSerifSC}

\definecolor{primary_color}{RGB}{0,51,153}
\definecolor{secondary_color}{RGB}{0,51,153}

\newlength{\iconwidth}
\setlength{\iconwidth}{1.5em}

%%%%%%%%%%%%%%%%%%%%
% 文章内容
%%%%%%%%%%%%%%%%%%%%

\newcommand{\school}{数学学院 | School of Mathematics}

%\newcommand{\contact}{
%	\small
%	\textcolor{black}{
%		\faEnvelope \quad \href{mailto:xxx@mails.jlu.edu.cn}{xxx@mails.jlu.edu.cn}
%		\hspace{4em}
%		\faPhone \quad  130-xxxx-xxxx
%		\hspace{4em}
%		\faGithub \quad \href{https://github.com/Reborn14}{GitHub 项目地址}
%	}
%}

\begin{document}
	
	%%%%%%%%%%%%%%%%%%%%
	% 页眉、页脚和背景
	%%%%%%%%%%%%%%%%%%%%
	
	\begin{tikzpicture}[remember picture, overlay]
		% 使用颜色矩形
		\node[anchor=north, inner sep=0pt, 
		minimum width=\paperwidth,      % 全纸张宽度
		minimum height=1.5cm,           % 调整高度
		fill=navyblue!80](header) at (current page.north)     % 深蓝色填充(自定义颜色)
		{};
		\node[anchor=west, inner sep=0pt, outer sep=0pt](school_logo) at (header.west){  % ← 这里添加 inner sep=0pt, outer sep=0pt
			\includegraphics[width=0.3\textwidth]{images/header_jlu.png}
		};
		\node[anchor=east](school_name) at(header.east){
			\textcolor{white}{\large\textbf{\school}}
			\hspace{0.5cm}
		};
	\end{tikzpicture}
	\vspace{-1.5em}
	
%	\begin{tikzpicture}[remember picture, overlay]
%		\node[anchor=south, inner sep=0pt](footer) at (current page.south){
%			\includegraphics[width=1.1\paperwidth, height=1cm]{images/foot.png}
%		};
%		\node[anchor=center] at(footer.center){\contact};
%	\end{tikzpicture}
	
%	\begin{tikzpicture}[remember picture, overlay]
%		\node[opacity=0.05] at(current page.center){
%			\includegraphics[width=0.7\paperwidth, keepaspectratio]{images/jlu_logo.png}
%		};
%	\end{tikzpicture}
%	
	%%%%%%%%%%%%%%%%%%%%
	% 简历正文
	%%%%%%%%%%%%%%%%%%%%
	
	\begin{minipage}[t]{0.78\textwidth}
		\begin{minipage}[t]{0.82\textwidth}
			\begin{minipage}[t]{\textwidth}
				\section[个人信息]{\normalsize\makebox[\iconwidth][c]{\color{primary_color}{\faAddressCard}}\quad 个人信息}
				% 使用 tabular 环境创建三行两列的布局
				\begin{tabular*}{\textwidth}{@{\extracolsep{\fill}}ll}
					\textbf{姓\qquad 名}:张三李四     & \textbf{性\qquad 别}:男/女 \\
					\textbf{出生年月}:xx.xx.xx    & \textbf{民\qquad 族}:xx \\
					\textbf{籍\qquad 贯}: xx省xx市      &\textbf{政治面貌}:xxx\\
					\textbf{邮\qquad 箱}:@mail.address & \textbf{电\qquad 话}:1234xxx \\
				\end{tabular*}
				\vspace{1.2em}
			\end{minipage}
		\end{minipage}
		
		\begin{minipage}[t]{\textwidth}
			\section[教育背景]{\normalsize\makebox[\iconwidth][c]{\color{primary_color}{\faGraduationCap}}\quad 教育背景}
			
			\begin{tabular}{
					@{}               
					l                
					@{\hspace{9.5em}} % 可调整间距
					l               
					@{\hspace{4.5em}} % 可调整间距
					r               
					@{}              
				}
				{\large \textbf{吉林大学}} & 本科 & xx年9月--x年6月 \\
			\end{tabular}
			\begin{itemize}
				\item 数学学院,xx学
				\item \textbf{主修课程}:数学分析、高等代数、xx专业课等。
				\item \textbf{GPA}:xx/ 4.0(排名:?/200)
			\end{itemize}
			
			\vspace{0.5em}
			
			\begin{tabular}{
					@{}
					l
					@{\hspace{9.5em}} % 可调整间距
					l
					@{\hspace{4.5em}} % 可调整间距
					r
					@{}
				}
				{\large \textbf{吉林大学}} & 硕士 & xx年9月--至今 \\
			\end{tabular}
			\begin{itemize}
				\item 数学学院,xx学
				\item \textbf{主修课程}:泛函分析、xx专业课等。
				\item \textbf{研究方向}:xx、xxx等。
			\end{itemize}
			
			\vspace{1em}
		\end{minipage}
		
	\end{minipage}
	\hfill
	\begin{minipage}[t]{0.2\textwidth}
		\vspace{2em}
		\setlength{\fboxsep}{0pt}
		\doublebox{\includegraphics[width=\linewidth]{images/avatar.png}}
	\end{minipage}
	
	 % 获奖情况
	 \section[获奖情况]{\normalsize\makebox[\iconwidth][c]{\color{primary_color}{\faStar}}\quad 获奖情况}
	 
	 % 荣誉1
	 {\large \textbf{硕士期间}} 
	 \begin{itemize}
	 	\item xx 奖学金,xx 奖学金,xx 竞赛一等奖。
	 	\item xx 称号。
	 \end{itemize}
	 
	 \vspace{0.5em} 
	 
	 % 荣誉2
	 {\large \textbf{本科期间}} 
	 \begin{itemize}
	 	\item xx 奖学金,xx 奖学金。
	 	\item xx 竞赛国家级x等奖、xx 竞赛省级x等奖。
	 	\item xx 项目结题。
	 \end{itemize}
	 
	
	\begin{minipage}[t]{\textwidth}
		\section[项目与教学]{\normalsize\makebox[\iconwidth][c]{\color{primary_color}{\faChalkboardTeacher}}\quad 项目与教学}
		
		% 项目1
		\begin{tabular}{
				@{}             
				l             
				@{\hspace{0.5em}} 
				l               
				@{\hspace{26em}} 
				r               
				@{}              
			}
			{\large \textbf{xx讨论班}} & (主讲/参与) & x年x月--至今 \\
		\end{tabular}
		\begin{itemize}
			\item \textbf{主要内容} 内容1;
			内容2;内容3。
		\end{itemize}
		
		\vspace{0.5em}
		
		% 项目2
		\begin{tabular}{
				@{}
				l
				@{\hspace{0.5em}}
				l
				@{\hspace{26em}}
				r
				@{}
			}
			{\large \textbf{xx项目}} & (负责人/参与) & x年x月--x年x月 \\
		\end{tabular}
		\begin{itemize}
			\item \textbf{项目简介}:本项目xx。
			\item \textbf{负责工作}:
			工作1;工作2。
		\end{itemize}
		
		\vspace{1em}
	\end{minipage}
	
	\begin{minipage}[t]{\textwidth}
		\section[技能特长]{\normalsize\makebox[\iconwidth][c]{\color{primary_color}{\faWrench}}\quad 技能特长}
		\begin{itemize}
			\setlength{\itemsep}{0.5em}
			\item 英语 CET-4、CET-6 等级考试。
			\item 熟练使用 xx、xx、xx等编程语言。
			\item 计算机几级证书
			\item xx几级证书
		\end{itemize}
	\end{minipage}

	
	% \newpage
	% % 如有需要,可以添加额外的页面。不要忘记添加页眉页脚和背景相关的代码。
	
	% % 竞赛经历
	% \section{\makebox[\widthof{\faTrophy}][c]{\color{primary_color}{\faTrophy}}\quad 竞赛经历}
	% \begin{table}[h!]
		%     \begin{tabularx}{\textwidth}{Xp{\widthof{第零负责人}}p{\widthof{国家级-第100名}}p{\widthof{2030年13月}}}
			%         \textbf{比赛1} & 第一负责人 & 国家级-第10名 & 2023年4月 \\
			%         \textbf{比赛2} & 个人参赛 & 国家级-一等奖 & 2023年8月\\
			%         \textbf{比赛3} & 个人参赛 & 省级-一等奖 & 2022年12月\\
			%         % 同理,可以自己加
			%     \end{tabularx}
		% \end{table}
	
	% % 技能特长
	% \section{\makebox[\widthof{\faWrench}][c]{\color{primary_color}{\faWrench}}\quad 技能特长}
	% \begin{itemize}
		%     \item 熟练使用\Cpp 、Python、Matlab编程语言。
		%     \item 熟悉Windows与Linux端开发。
		%     \item 熟练使用Tensorflow,Pytorch等深度学习框架。
		%     \item 熟练掌握\Cpp 与Python环境下OpenCV与Qt应用的开发,且熟练使用Qt Creator软件。
		%     \item 熟练使用Altium Designer与LCEDA进行封装绘制与板子设计。
		%     \item 熟练使用Keil,Arduino IDE等集成开发软件。
		%     \item 了解模式识别,强化学习,遗传算法,知识蒸馏等相关概念。
		% \end{itemize}
	
	% % 所获荣誉
	% \section{\makebox[\widthof{\faStar}][c]{\color{primary_color}{\faStar}}\quad 所获荣誉}
	% \begin{multicols}{2}
		%     \begin{itemize}
			%         \item 某年学业先进个人
			%         \item 某年某奖学金某等奖
			%         \item 某大使
			%         \item 某年某奖学金某等奖
			%         \item 某年优秀团员称号
			%         \item 某年某称号
			%     \end{itemize}
		% \end{multicols}
	
	% % 其他
	% \section{\makebox[\widthof{\faInfo}][c]{\color{primary_color}{\faInfo}}\quad 其他}
	% \begin{itemize}
		%     \item 英语水平-CET6级xxx分
		%     \item 计算机几级证书
		%     \item xx几级证书
		%     \item 技术博客: 某网址
		%     \item 教师资格证:xxx
		%     \item 普通话证书:几级几等
		%     \item 文字排版:\LaTeX
		% \end{itemize}
	
\end{document}